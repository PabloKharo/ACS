\documentclass{article}
\usepackage[utf8]{inputenc}
\usepackage[russian]{babel}


\title{\textbf{Лабораторная работа №2}}
\usepackage{sagetex}
\setlength{\sagetexindent}{10ex}

\usepackage{geometry}
\geometry{a4paper, total={170mm,257mm}, left=20mm, right=20mm, top=20mm, bottom=20mm}

\begin{document}

\maketitle

\section*{Задание 3. Привести уравнение к каноническому виду}
\textbf{Определяем переменные \textit{x, y, z} и функцию \textit{func} моего варианта.}
\begin{sageblock}
var("x y z")
func = 11*x**2 - 2*x*y - 2*x*z + 2*y*z + 9*z**2 - 4*x + y + z
\end{sageblock}
\textbf{График данной функции:} 
\begin{sageblock}
plot_start = implicit_plot3d(func, (x,-5,5), (y,-5,5), (z,-5,5), plot_points=100)
\end{sageblock}
\sageplot{plot_start}

\textbf{Составляем матрицу квадратичной формы \textit{A} , столбец коэффициентов линейной формы \textit{a} и единичную матрицу \textit{E}.}

\begin{sageblock}
A = matrix([[11, -1, -1], [-1, 0, 1], [-1, 1, 9]])
a = matrix([[-2], [1/2], [1/2]])
E = matrix([[1, 0, 0], [0,1,0], [0,0,1]])
\end{sageblock}

\textbf{Находим корни характеристического уравнения.}
\begin{sageblock}
var("l l1 l2 l3")
lvct = vector([l1, l2, l3])
lambdas = solve(det(A-l*E), l)
\end{sageblock}


\textbf{Находим взаимно перпендикулярные единичные собственные векторы \textit{s1, s2, s3}, соответствующие корням \textit{l1, l2, l3}.}
\begin{sageblock}
svcts = []
for el in lambdas:
    nums = []
    lhs = (A-el.rhs()*E)*lvct
    res = solve([lhs[0] == 0, lhs[1] == 0, lhs[2] == 0], l1, l2, l3)[0]   
    for i in range(len(res)):
        if len(res[i].rhs().variables()) == 0:
            nums.append(res[i].rhs())
        else:
            varname = res[i].rhs().variables()[0]
            d = {varname: 1}
            nums.append(res[i].rhs().subs(d).n())
    svcts.append(vector(nums))
    
for i in range(len(lambdas)):
    lambdas[i] = real_part(lambdas[i].rhs().n())
\end{sageblock}

\textbf{Нормируем эти векторы.}
\begin{sageblock}
normvcts = []
for el in svcts:
    norm_lvct = (el / sqrt((el*el).n())).n()
    normvcts.append(norm_lvct)

\end{sageblock}

\textbf{Получаем матрицу $S^T$ и получаем $a' = S^T*a$.}
\begin{sageblock}
ST = matrix([[0, 0, 0],[0, 0, 0],[0, 0, 0]])
for i in range(len(normvcts)):
    for j in range(len(normvcts[i])):
        ST[i,j] = normvcts[i][j]

a_ = ST*a
show("a\': ", tuple(a_))
\end{sageblock}

\textbf{Так как \textit{a'} = 0, то получаем канонический вид нашей функции.}
\begin{sageblock}
kanon_func = lambdas[0]*x^2 + lambdas[1]*y^2 + lambdas[2]*z^2
show(kanon_func)
\end{sageblock}
\begin{center}
$\sage{lambdas[0]}*x^2 \sage{lambdas[1]}*y^2 + \sage{lambdas[2]}*z^2$
\end{center}

\textbf{График функции, приведенной к каноническому виду:} 
\begin{sageblock}
plot_end = implicit_plot3d(kanon_func, (x,-5,5), (y,-5,5), (z,-5,5), plot_points=100)
\end{sageblock}
\sageplot{plot_end}

\end{document}